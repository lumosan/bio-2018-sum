%!TEX root = report.tex
\documentclass[a4paper]{article}
\input{_core.tex}

\usepackage{parskip}

\addbibresource{mybib.bib}

% Global variables
  \newcommand{\HWTitle}{Biometrics}
  \newcommand{\HWSubtitle}{Summary}
  \newcommand{\HWDueDate}{\today}
  \newcommand{\HWAuthorName}{Lucía Montero Sanchis}
  \title{\vspace{-.25cm} \HWTitle \\ \vspace{.25cm}}
  %\date{\HWDueDate}
  \author{\HWAuthorName}

\begin{document}
\maketitle

\section*{Question 1 -- Leading Biometric Technology} % (fold)
\label{sec:question_1}
  \subsection*{Physiological Characteristics}
    \input{fgfci}
    \input{perih}
  \subsection*{Behavioral Characteristics}
    \input{vdyn}
    \subsubsection*{09. Gait, Typing Rhythm}
      \textbf{Gait}: How people walk
      \begin{itemize}
        \item \textbf{Generalities}: Walking is similar for all humans. Non-contact. Uses sequences. Applications: security/surveillance, medicine, forensics. Hard to disguise and perceivable at distance.
        \item \textbf{Features}: Width of silhouette; vertical and horizontal projections; angular representation; PCA and LDA.
        \item \textbf{Feature extraction}: Recognition includes \emph{dynamic} (motion) and \emph{static} (body shape) features.
        \begin{enumerate}
          \item Global \emph{motion} and \emph{shape}
          \item Gait \emph{period} and \emph{phase}
          \item Gait model initialisation
          \begin{itemize}
            \item Extended pendular thigh-model based on angles
            \item Forced oscillator/bilateral symmetry/phase coupling
          \end{itemize}
          \item Local contour deformation
        \end{enumerate}
        \item \textbf{Matching}: DTW and HMM
      \end{itemize}

      \textbf{Typing Rhythm} (keystroke dynamics):
      \begin{itemize}
        \item \textbf{Generalities}: Way a user types on a keyboard, identity verification is based on how you type. Behavioral, transparent and natural. Can be used on smartphones. Typing pattern may be unique because it's based on neuro-physiological factors (similar to dynamic signature)\\
        Narrow range of applications, need to account for typing errors. Minimal training, no additional hardware.
        \item \textbf{Features}:
        \begin{itemize}
          \item \textbf{Latencies} between succesive keystrokes
          \item \textbf{Duration} of each keystroke
          \item Overall typing speed
          \item Especially consistent features for known regularly-typed strings (e.g. username and password)
        \end{itemize}
        \item \textbf{Trigraph features}: 3 consecutively typed keys. \emph{Duration}: (time between first and third key). Feature vector in trigraphs are sorted in ascending duration order.
        \item \textbf{Trigraph matching}: \emph{Degree of disorder} (sum of absolute changes in position between two sorted arrays, normalized to have a value in [0,1])
        \item\textbf{Password hardening} -- Text password + Typing pattern\\
        More security, but also high false rejects (e.g. change of keyboard).
      \end{itemize}
  \subsection*{Biological Traces}
    \subsubsection*{10. DNA}
      \textbf{Generalities}:
      \begin{itemize}
        \item \emph{Biological traces}
        \item DNA \textbf{unique} to every individual, only shared by \textbf{identical twins}. Can be obtained from many sources. Does not change during life, but can be damaged by chemical and physical events and by random mutations.
        \item Invasive collection of samples. Money and time expensive. Privacy concerns?
        \item Requires \textbf{tangible} physical sample and matching is not done in real-time. Currently not completely automated.
        \item \emph{Biological background}: All cells of an organism contain same DNA content. A \textbf{chromosome} is the visible state of genetic material. Humans have 23 pairs of chromosomes. Genes are made of \textbf{DeoxyriboNucleic Acid} (DNA).\\
        \textbf{DNA}: Double helix is the natural form in which DNA is found within nucleus of the cells. DNA is a polymer (long string of simple repeating units). Repeating units are called \textbf{nucleotides} (four types: \textbf{A}denine, \textbf{C}ytosine, \textbf{G}uanine, \textbf{T}hymine). The complete DNA molecule consists of two of these strands of the four bases.
        \item Components:
        \begin{itemize}
          \item Nucleotides
          \item Hydrogen bonds (electrostatic connection): A-T; C-G
        \end{itemize}
      \end{itemize}

      \textbf{Concepts}:
      \begin{itemize}
        \item Most DNA is the same accross people, but some varies: \textbf{Short Tandem Repeats (STRs)} in non-coding sequences.
        \item \textbf{Polymerase Chain Reaction (PCR)} -- Method of amplifying a specific region of the genome (from 1 to over 1 Billion copies)
        \begin{itemize}
          \item \textbf{Locus} -- region of the genome being examined \todo[inline]{??}
          \item \textbf{Allele} -- State of the genetic variation being examined. STRs is the nb. of repeat units, SNPs is the base sequence at the site.
          \item Chromosomes are \textbf{paired}: Homozygous (Heterozygous) -- Alleles identical (differ) on each chromosome
          \item \textbf{Process}:
          \begin{enumerate}
            \item DNA molecule with target sequence is heated to denature it
            \item When the mixture cools, primers bond to the single stranded DNA
            \item dNTPs and DNA polymerase are added to synthesize two new strands of DNA
            \item Process is repeated, many copies can be produced in a short time
          \end{enumerate}
        \end{itemize}
        \item \textbf{Short Tandem Repeats (STRs)} -- Length Variation\\
        These repeat regions usually bounded by specific \emph{restriction enzyme} sites $\Rightarrow$ possible to cut out segment of chromosome
        \item \textbf{Single Nucleotide Polymorphisms (SNPs)} -- Sequence Variation; insertions/deletions
        \item \textbf{Capillary Electrophoresis} -- Separates the fragments by drawing them towars a positively charged pole. Shortest fragments move faster, therefore their order reflects their size. Laser light activates the fluorescent tags as the fragments pass a detection window, producing a color readout that is translated into a sequence.
        \todo[inline]{ddNTP? Locus? watch a video on this whole thing? :/}
      \end{itemize}

      \textbf{DNA Analysis steps}:
      \begin{enumerate}
        \item Collection: Blood, saliva, semen, urine, hair, teeth, bone, tissue
        \item Specimen Storage
        \item Extraction
        \item Quantitation
        \item Genotyping
        \item Interpretation of results
        \item Database (storage and searching)
      \end{enumerate}

      \textbf{Applications}:
        Forensics, Paternity testing, Historical investigations, Missing persons investigations, Mass disasters
  \subsection*{Synthetic Biometric Data Generation}
    \subsubsection*{11. Synthesis of Fingerprints}
      \textbf{Synthetic Fingerprint Generation}: To automatically create large databases of fingerprints, allowing to train, test, optimize and compare algorithms. Collecting fingerprints is expensive (money and time) and problematic (privacy legislation).
      \begin{enumerate}
        \item Select  $\rightarrow$ Obtain:
        \begin{enumerate}
          \item shape parameters $\rightarrow$ \textbf{fingerprint shape}
          \item class and singularities $\rightarrow$ \textbf{directional map} model
          \item average density (and singularities) $\rightarrow$ \textbf{density map} model
        \end{enumerate}
        \item \textbf{Ridge pattern} generation, to obtain \textbf{master} fingreprint
        \begin{itemize}
          \item Gabor filters iteratively applied to an initially white image, enriched with a few random points. Orientation and frequency of the filters are locally adjusted according to directional and density maps.
          \item As a result, realistic minutiae appear at random positions.
        \end{itemize}
        \item Add \textbf{variability} data:
        \begin{enumerate}
          \item Determine contact region and erosion/dilation (low-pressure or dry/high-pressure or wet)
          \item Skin deformation
          \item Noise and rendering
          \item Translation and rotation
          \item Generate background
        \end{enumerate}
      \end{enumerate}
      \begin{figure}[htp]
        \centering
          \includegraphics[width=.45\textwidth]{fg.png}
          \caption{Synthetic Fingerprint Generation}
          \label{fig:fg}
      \end{figure}
  \newpage
  \subsection*{Multimodal Biometrics}
    \subsubsection*{12. Multimodal Biometrics}
      \textbf{Generalities}:
      \begin{itemize}
        \item \emph{Desired characteristics}: Universality; Uniqueness; Permanence; Collectability; Performance (achievable recognition accuracy, resources required, operating environment); Acceptability; Circumvention (how easily can it be spoofed?)
        \item Every biometric characteristic has some limitations $\Rightarrow$ \emph{Solution}: \emph{Multimodal Biometrics}
        \item \textbf{Multimodality} (combine evidence from multiple traits)
        \begin{itemize}
          \item High cost; Longer verification time
          \item Permit choice of biometric; Increase population coverage, reduce failure to enroll; Enhance performance; Improve resilience to spoofing
        \end{itemize}
      \end{itemize}

      \textbf{Classification of biometrics}:
      \begin{itemize}
        \item \textbf{Unimodal} (\emph{most restrictive}) -- subset of a unibiometric system that uses a single instance (snapshot), a single representation and a single matcher for recognition
        \item \textbf{Unibiometric} -- uses a single biometric identifier
        \item \textbf{Multi-biometric} -- uses more than one independent biometric identifier (e.g. fingerprint and face)
        \item \textbf{Multimodal} (\emph{most general}) -- superset of a multi-biometric system that may use more than one correlated biometric measurement (e.g. multiple impressions of a finger)
      \end{itemize}

      \textbf{What to integrate?}:
        \begin{itemize}
          \item \textbf{Multiple sensors} for same biometric (e.g. optical and solid-state sensors for fingerprints)
          \item \textbf{Multiple biometrics (traits)} -- Each sensor senses a different biometric. Improves system accuracy and matching speed
          \item \textbf{Multiple samples} of same biometric, e.g. multiple impressions of the same finger
          \item \textbf{Multiple instances}: e.g. fingerprints from two or more fingers.
          \item \textbf{Multiple representations} and matching algorithms for the same biometric -- combines different approaches to feature extraction and matching of single biometric.
        \end{itemize}

      \textbf{Integration strategies}:
        \begin{itemize}
          \item \textbf{Integration architecture} -- Combination of classifiers:
          \begin{itemize}
            \item Parallel
            \begin{itemize}
              \item Early integration (\emph{Feature} extraction)
              \item Late integration (\emph{Matching Score} or \emph{Decision})
            \end{itemize}
            \item Serial
            \item Hierarchical
          \end{itemize}
          \item \textbf{Levels of fusion}:
          \begin{itemize}
            \item \textbf{Before matching} (\emph{often not possible}):
            \begin{itemize}
              \item \textbf{Data/Sensor Level} (combine raw sensor outputs) -- Mosaicing for fingerprints, 2-3D faces
              \item \textbf{Feature Level} (combine extracted features)
            \end{itemize}
            \item \textbf{After matching}:
            \begin{itemize}
              \item Dynamic Classifier Selection
              \item \textbf{Classifier Fusion}:
              \begin{itemize}
                \item \textbf{Score Level} (combine matching scores, \emph{most popular}, \emph{Classification} or \emph{Combination} approaches possible)
                \item \textbf{Rank Level} -- Consolidates the ranks associated with every subject. Highest rank (best rank assigned to an individual by any of the classifiers); Borda count (number of identities whose ranks are worse than the individual's rank)
                \item \textbf{Decision Level} (combine identity decisions, \emph{least informative}). Possible schemes: AND, OR, majority voting\\
                \emph{Support Vector Machines} find the hyperplane that gives the largest minimum distance to the training samples
              \end{itemize}
            \end{itemize}
          \end{itemize}
          \item \textbf{Fusion strategy}
          \begin{itemize}
            \item \emph{Quality based}: Multi-classifier Stacking (1 signal, several features sets extracted), Multi-biometrics Stacking (several signals)
            \item \emph{Fusion after normalization} (Score normalization, weighted sum, decimal scaling, Z score, double sigmoid function...)
          \end{itemize}
        \end{itemize}
  \subsection*{Miscellaneous}
    \subsubsection*{13. Quality and Ageing in Classication of Biometric Data}

\newpage

\section*{Question 2 -- Other Topics} % (fold)
\label{sec:question_2_other_topics}
  \subsection*{Fundamentals of Biometrics (01)}
    \input{idvauth}
  \subsection*{Analysis, Modeling and Interpretation of Biometric Data}
    \input{dtwgmmhmmpca}
    \subsubsection*{08. Enrollment and Template Creation}
    \subsubsection*{09. Verification and Identification System Errors}
    \subsubsection*{10. Evaluation of Biometric Systems (01,03)}
      \textbf{Performance evaluation}:
      \begin{itemize}
        \item \textbf{Evaluation factors}: Speech quality, Speech modality, Speech duration, Speaker population.
        \item \textbf{FMR} and \textbf{FNMR}:
        \begin{itemize}
          \item False Match Rate \textbf{FMR} -- proportion of impostor attempt samples falsely declared to match the compared nonself template (number of impostor FMs / number of impostor attempts)
          \item False Non-Match Rate \textbf{FNMR} -- proportion of genuine attempt samples falsely declared not to match the template of the same characteristic from the same user submitting the sample (number of genuine FNMs / number of genuine attempts)
          \item \textbf{Calculation}: Create biometric templates using training data set. Define a test set with genuine and impostor trials. Run test and group genuine and impostor scores. Choose threshold value T and calculate FMR(T) and FNMR(T).
        \end{itemize}
        \item \textbf{FAR} and \textbf{FRR}:
        \begin{itemize}
          \item False Accept Rate \textbf{FAR} -- Proportion of imposters accepted (security breaches)\\\textbf{FAR}=FMRx(1-FTA)
          \item False Reject Rate \textbf{FRR} -- Proportion of genuine users rejected (inconvenience)\\\textbf{FRR}=FTA+FNMRx(1-FTA)
        \end{itemize}
        \item \textbf{DET} and \textbf{ROC} curves:
        \begin{itemize}
          \item Detection Error Tradeoff \textbf{DET} -- can be computed from distributions of scores with a variable threshold. FAR vs FRR
          \item Receiver Operating Characteristic \textbf{ROC} -- Correct Accept Rate as function of False Accept Rate (FAR)
        \end{itemize}
        \item \textbf{CAR} and \textbf{CRR}:
        \begin{itemize}
          \item Convenience -- Correct Accept Rate \textbf{CAR}=1-FRR
          \item Security -- Correct Reject Rate \textbf{CRR}=1-FAR
        \end{itemize}
        \item \textbf{FTA} and \textbf{FTE}:
        \begin{itemize}
          \item Failure to Acquire Rate \textbf{FTA} -- Proportion that cannot be verified (does not process a certain biometric)
          \item Failure to Enroll Rate \textbf{FTE} -- Proportion that cannot be enrolled (system fails to complete the enrolment process, due to bad quality)
        \end{itemize}
        \item \textbf{Performance measures for identification}:
        \begin{itemize}
          \item Correct Identification Rate (\textbf{CIR}) -- proportion of identification transactions by users in the system s.t. the user's identifier is among the ones returned.
          \item Cumulative Match Characteristic (\textbf{CMC}) -- Identification rate as function of K.
        \end{itemize}

      \end{itemize}
      \begin{figure}[htp]
        \centering
          \includegraphics[width=.5\textwidth]{thres.png}
          \caption{Threshold T, FRR and FAR}
          \label{fig:frrfar}
      \end{figure}
    \newpage
  \subsection*{Miscellaneous}
    \input{fspk}
    \subsubsection*{12. Forensic Biometrics (Fingerprints, Face, DNA, Ear, Gait)}
      \begin{itemize}
        \item \textbf{Fingerprints}: $$LR = \frac{p(\text{evidence}|H_p)}{p(\text{evidence}|H_d)}$$
        with $H_p$ (suspect left the fingerprint), $H_d$ (someone else left the fingerprint), numerator (variability of minutiae configurations due to distortion and clarity), denominator (variability between minutiae from different sources). Delaunay triangulation, MCC and Local Quality Measures (embedding quality measures), ridge quality maps are used.
        \item \textbf{Face}:
        \begin{itemize}
          \item Challenges: Forensic Sketch Recognition, Facial Aging, Facial Marks, Unconstrained Facial recognition
          \item De-identification: Blurring, pixelation
          \item Near infra-red to Visible Light images
        \end{itemize}
        \item \textbf{DNA}: Very used in forensics to match crime scene evidence to individuals.
        \begin{itemize}
          \item DNA profiles from a single region (Locus); DNA lineup of the \emph{suspects}
          \item Uses: Identify a person; exclude a suspect; link suspect, victim and crime scene; link weapon to victim; link witness to scene; (dis)prove alibi; reconstruct scene; provide investigative leads.
          \item \emph{Innocence Project} -- To exonerate falsely incarcerated individuals.
          \item Missing persons investigation and mass disasters -- Only possible method to identify remains in some cases. Time is the biggest enemy. Family members can be used to identify remains.
        \end{itemize}
        \item \textbf{Ear}: Identify ear traces
        \item \textbf{Gait}: How people walk. Identification of burglars from security/surveillance camera recordings.
      \end{itemize}
    \subsubsection*{13. Biometric Standards}
    \subsubsection*{14. Securing Biometric Data and Biometric Encryption}
    \subsubsection*{15. Biometrics in Identity Documents}
    \subsubsection*{16. Privacy and Legal Issues}

\end{document}





\begin{figure}[htp]
  \centering
    % \includegraphics[width=\textwidth]{images/figure.pdf}
    \missingfigure[figwidth=\textwidth]{Some Figure}
    \caption{Sample Figure}
    \label{fig:sample fig}
\end{figure}

\begin{figure}[htp]
  \centering
  \begin{subfigure}{.45\textwidth}
    \centering
    \caption{First Subfigure}
    % \includegraphics[width=\textwidth]{images/figure.pdf}
    \missingfigure[figwidth=\textwidth]{Some subfigure}
    \label{fig:samplesubfiga}
  \end{subfigure}%
  \hfill
  \begin{subfigure}{.45\textwidth}
    \centering
    \caption{Second Subfigure}
    % \includegraphics[width=\textwidth]{images/figure.pdf}
    \missingfigure[figwidth=\textwidth]{Some subfigure}
    \label{fig:samplesubfigb}
  \end{subfigure}
  \caption{Sample Figure with subfigures}
  \label{fig:samplesubfig}
\end{figure}

\begin{table}[htp]
  \centering
  \renewcommand{\arraystretch}{1.2}
  \begin{tabular}{llr}
    \toprule
    \multicolumn{2}{c}{Item} & \multirow{2}{*}{Price (\$)} \\
    \cmidrule(r){1-2}
    Animal    & Description &   \\
    \midrule
    \multirow{2}{*}{Gnat}      & per gram    & 13.65      \\
              &    each     & 0.01       \\
    Gnu       & stuffed     & 92.50      \\
    Emu       & stuffed     & 33.33      \\
    Armadillo & frozen      & 8.99       \\
    \bottomrule
  \end{tabular}
  \caption{Some table }
  \label{tab:sample table}
\end{table}
